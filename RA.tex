\documentclass[11pt]{article}
\usepackage{lmodern}
\usepackage{geometry}
\usepackage{fontspec}
\usepackage{titlesec}
\usepackage{amsmath}
\usepackage{fancyhdr}
\pagestyle{plain}
\usepackage{amssymb}
\usepackage{ntheorem}
\usepackage{bm}
\usepackage{graphicx}
\usepackage{braket}
\usepackage{hyperref}
\usepackage{mathrsfs}
\theorembodyfont{\normalfont}
\newtheorem{theorem}{Theorem}[subsection]
\newtheorem{definition}[theorem]{Definition}
\newtheorem{proposition}[theorem]{Proposition}
\newtheorem{corollary}[theorem]{Corollary}
\newtheorem{lemma}[theorem]{Lemma}

\title{\textsc{Real Analysis: Definitions and Theorems}}
\author{\textsc{Jincheng Wang}}
\date{\url{jc-wang@sjtu.edu.cn}}
\begin{document}
\maketitle
\section{Measure Theory}
\section{Lebesgue Integral}
\section{Differentiation and Integral}
\section{Hilbert Space: An Introduction}
\subsection{$L^2$ space}
\begin{proposition}
    The Space $L^2(\mathbb{R}^d)$ has the following properties:
    \begin{description}
        \item[(i)] $L^2(\mathbb{R}^d)$ is a vector space.
        \item[(ii)]$f(x)\overline{g(x)}$ is integrable whenever $f,g\in L^2(\mathbb{R}^d)$, and the Cauchy-Schwarz inequality holds: $|(f,g)|\leq ||f||\ ||g||$.
        \item[(iii)]  If $g\in L^2(\mathbb{R}^d)$ is fixed, the map $f\mapsto (f,g)$ is linear in $f$, and also $(f,g)=\overline{(g,f)}$.
        \item[(iv)] The triangle inequality holds: $||f+g||\leq||f||+||g||$
    \end{description}
\end{proposition}

\begin{theorem}
    The space $L^2(\mathbb{R}^d)$ is complete in its metric.
\end{theorem}

\begin{theorem}
    The space $L^2(\mathbb{R}^d)$ is \textbf{separable}, int the sense that there exists a countable collection $\{f_k\}$ of elements in $L^2(\mathbb{R}^d)$ such that their linear combinations are dense in $L^2(\mathbb{R}^d)$
\end{theorem}
\subsection{Hilbert space}
\begin{definition}
    A set $\mathcal{H}$ is a \textbf{Hilbert Space} if it satisfies the following:
    \begin{description}
        \item[(i)] $\mathcal{H}$ is a vector space over $\mathbb{C}$ (or $\mathbb{R}$).
        \item[(ii)] $\mathcal{H}$ is equipped with an inner product $(\cdot,\cdot)$, so that
        \begin{description}
            \item[1.] $f\mapsto (f,g)$ is linear on $\mathcal{H} $ for every fixed $g\in \mathcal{H} $
            \item[2.] $(f,g)=\overline{(g,f)}$
            \item[3.] $(f,f)\geq 0$ for all $f\in \mathcal{H} $ 
        \end{description}
        We let $||f||=(f,f)^{1/2}$.
        \item[(iii)] $||f||=0$ if and only if $f=0$.
        \item[(iv)] The Cauchy-Schwarz and triangle inequalities hold \[|(f,g)|\leq ||f||\ ||g||\quad and\quad ||f+g||\leq||f||+||g||\] 
        \item[(v)] $\mathcal{H} $ is complete in the metric $d(f,g)=||f-g||$.
        \item[(vi)] $\mathcal{H} $ is seperable.
    \end{description}
\end{definition}

\begin{definition}
    \textbf{(Orthogonality)}
    Two element $f$ and $g$ in a Hilbert space $\mathcal{H}$ with inner product $(\cdot,\cdot)$ are \textbf{orthogonal} or \textbf{perpendicular} if $(f,g)=0$, and we write $f\perp g$.
\end{definition}

\begin{proposition}
    If $f\perp g$, then $||f+g||^2=||f||^2+||g||^2$.
\end{proposition}

\begin{proposition}
    If $\{e_k\}_{k=1}^\infty$ is orthonormal, and $f=\sum a_ke_k\in \mathcal{H}$ where the sum is finite, then
    \[||f||^2=\sum|a_k|^2.\]
\end{proposition}

\begin{theorem}
    The following properties of an orthonormal set $\{e_k\}_{k=1}^\infty$ are equivalent.
    \begin{description}
        \item[(i)] Finite linear combinations of elements in $\{e_k\}$ are dense in $\mathcal{H}$.
        \item[(ii)] If $f\in\mathcal{H}$ and $(f,e_j)=0$ for all $j$, then $f=0$.
        \item[(iii)] If $f\in\mathcal{H}$, and $S_N(f)=\sum_{k=1}^Na_ke_k$, where $a_k=(f,e_k)$, then $S_N(f)\to f$ as $N\to \infty$ in the norm.
        \item[(iv)] If $a_k=(f,e_k)$, then $||f||^2=\sum_{k=1}^\infty |a_k|^2$ 
    \end{description}
\end{theorem}

\begin{theorem}
    Any Hilbert space has an orthonormal basis.
\end{theorem}

\begin{definition}
    Give two Hilbert spaces $\mathcal{H}$ and $\mathcal{H}'$ with respective inner products $(\cdot,\cdot)_\mathcal{H}$ and $(\cdot,\cdot)_{\mathcal{H}'}$. A mapping $U:\mathcal{H}\to\mathcal{H}'$ between these space is called \textbf{unitary} if:
    \begin{description}
        \item[(i)] $U$ is linear, that is, $U(\alpha f+\beta g)=\alpha U(f)+\beta U(g)$.
        \item[(ii)] $U$ is a bijection.
        \item[(iii)] $||Uf||_{\mathcal{H}'}=||f||_{\mathcal{H}}$ for all $f\in \mathcal{H}$
    \end{description}
\end{definition}

\begin{corollary}
    Any two infinte-dimesional Hilbert spaces are unitarily equivalent.
\end{corollary}

\begin{corollary}
    Any two finite-dimensional Hilbert spaces are unitarily equivalent if and only if they have the same dimension.
\end{corollary}

\begin{definition}
    \textbf{Pre-Hilbert space} is a space $\mathcal{H}_0$ that satisfies all the defining properties of a Hilbert space except (v).
\end{definition}

\begin{proposition}
    Suppose we are given a pre-Hilbert space $\mathcal{H}_0$ with inner product $(\cdot,\cdot)_0$. Then we can find a Hilbert space $\mathcal{H}$ with inner product $(\cdot,\cdot)$ such that
    \begin{description}
        \item[(i)] $\mathcal{H}_0\subset \mathcal{H}$.
        \item[(ii)] $(f,g)_0=(f,g)$ whenever $f,g\in\mathcal{H}_0$.
        \item[(iii)] $\mathcal{H}_0 $ is dense in $\mathcal{H}$.
    \end{description}
\end{proposition}

\subsection{Fourier series and Fatou's theorem}
\begin{theorem}
    Suppose $f$ is integrable on $[-\pi,\pi]$.
    \begin{description}
        \item[(i)] If $a_n=0$ for all $n$, then $f(x)=0$ for a.e. x.
        \item[(ii)] $\sum_{n=-\infty}^\infty a_nr^{|n|}e^{inx}$ tends to $f(x)$ for a.e. x, as $r\to1 $, $r<1$.
    \end{description}
    In the theorem above, $a_n$ is the n-th Fourier coefficient of $f$
    \[a_n=\frac{1}{2\pi}\int_{-\pi}^{\pi}f(x)e^{-inx}dx\]
\end{theorem}
\begin{theorem}
    Suppose $f\in L^2([-\pi,\pi])$. Then:
    \begin{description}
        \item[(i)] We have Parseval's relation \[\sum_{n=-\infty}^\infty|a_n|^2=\frac{1}{2\pi}\int_{-\pi}^\pi |f(x)|dx\].
        \item[(ii)] The mapping $f\mapsto \{a_n\}$ is a unitary correspondence between $L^2([-\pi,\pi])$ and ${l}^2(\mathbb{Z})$.
        \item[(iii)] The Fourier series of $f$ converges to $f$ in the $L^2$-norm, that is,\[\frac{1}{2\pi}\int_{-\pi}^\pi|f(x)-S_N(f)(x)|^2dx\to 0\quad as\ N\to\infty\] where $S_N(f)=\sum_{|n|\leq N}a_ne^{inx}$.
    \end{description}
\end{theorem}

\begin{definition}
    If $F$ is a function defined in the unit disc $\mathbb{D}$, we say that $F$ has a radial limit at the point $-\pi\leq\theta\leq\pi$ on the circle, if the limit\[\lim_{r<1,\ r\to 1}F(re^{i\theta})\] exists.
\end{definition}

\begin{theorem}
    A bounded holomorphic function $F(re^{i\theta})$ on the unit disc has radial limits at almost every $\theta$.
\end{theorem}

\begin{definition}
    We define the \textbf{Hardy Space} $H^2(\mathbb{D})$ to consist of all holomorphic functions $F$ on the unit disc $\mathbb{D}$ that satisfy
    \[\sup_{0\leq r<1}\frac{1}{2\pi}\int_{-\pi}^{\pi}|F(re^{i\theta})|^2d\theta<\infty\]
    we also define the "norm" for functions $F$ in this class, $||F||_{H^2(\mathbb{D})}$, to be the square root of the above quantity.
\end{definition}

\subsection{Closed subspaces and orthogonal projections}
\begin{definition}
    A \textbf{linear subspace} $\mathcal{S}$ of $\mathcal{H}$ is a subset of $\mathcal{H}$ that satisfies $\alpha f+\beta g\in\mathcal{S}$ whenever $f,g\in \mathcal{S}$ and $\alpha$, $\beta$ are scalars. The subspace $\mathcal{S}$ is \textbf{closed} if whenever $\{f_n\}\subset \mathcal{S}$ converges to some $f\in \mathcal{H}$, then $f$ also belongs to $\mathcal{S}$.
\end{definition}

\begin{lemma}
    Suppose $\mathcal{S}$ is a closed subspace of $\mathcal{H}$ and $f\in \mathcal{H}$. Then:
    \begin{description}
        \item[(i)] There exists a (unique) element $g_0\in\mathcal{S}$ which is closest to $f$, in the sense that\[||f-g_0||=\inf_{g\in\mathcal{S}}||f-g||\]
        \item[(ii)] The element $f-g_0$ is perpendicular to $\mathcal{S}$, that is,\[(f-g_0,g)=0\quad for\ all\ g\in \mathcal{S}\]
    \end{description}
\end{lemma}

\begin{definition}
    If $\mathcal{S}$ is a subspace of a Hilbert space $\mathcal{H}$, we define the orthogonal complement of $\mathcal{S}$ by\[\mathcal{S}^\perp=\{f\in\mathcal{H}:(f,g)=0\quad for\ all\ g\in\mathcal{S}\}\]
\end{definition}

\begin{proposition}
    If $\mathcal{S}$ is a closed subspace of a Hilbert space $\mathcal{H}$, then
    \[\mathcal{H}=\mathcal{S}\bigoplus \mathcal{S}^\perp\]
\end{proposition}

\begin{definition}
    The mapping $P_\mathcal{S}$ is called the orthogonal projection onto $\mathcal{S}$ and satisfies the following simple properties:
    \begin{description}
        \item[(i)] $f\mapsto P_\mathcal{S}(f)$ is linear.
        \item[(ii)] $P_\mathcal{S}(f)=f$ whenever $f\in\mathcal{S}$.
        \item[(iii)] $P_\mathcal{S}(f)=0$ whenever $f\in\mathcal{S}^\perp$.
        \item[(iv)] $||P_\mathcal{S}(f)||\leq||f||$ for all $f\in \mathcal{H}$.
    \end{description}
\end{definition}

\subsection{Linear transformation}
\begin{definition}
    Suppose $\mathcal{H}_1$ and $\mathcal{H}_2$ are two Hilbert spaces. Amapping $T:\mathcal{H}_1\to\mathcal{H}_2$ is a linear transformation (also called linear operator or operator) if
    \[T(af+bg)=aT(f)+bT(g)\quad for\ all\ scalars\ a,b\ and\ f,g\in \mathcal{H}_1\]
    We can also say that a linear operator $T:\mathcal{H}_1\to\mathcal{H}_2$ us bounded if there exists $M>0$ so that
    \[||T(f)||_{\mathcal{H}_2}\leq M||f||_{\mathcal{H}_1}\]
    The norm of $T$ is denoted by $||T||_{\mathcal{H}_1\to \mathcal{H}_2}$ and defined by
    \[||T||=\inf M\]
\end{definition}

\begin{lemma}
    $||T||=\sup\{|(Tf,g)|:\ ||f||\leq 1,\ ||g||\leq 1\}$, where of course $f\in \mathcal{H}_1$ and $g\in \mathcal{H}_2$.
\end{lemma}

\begin{definition}
    A linear transformation $T$ is continous if $T(f_n)\to T(f)$ whenever $f_n\to f$.
\end{definition}

\begin{proposition}
    A linear operator $T:\mathcal{H}_1\to\mathcal{H}_2$ is bounded if and only if it is continuous.
\end{proposition}

\begin{definition}
    A \textbf{linear functional} $l$ is a linear transformation from Hilbert space $\mathcal{H}$ to the underlying field of scalars, which we may assume to be the complex numbers,
    \[l:\mathcal{H}\to\mathbb{C}\]
\end{definition}

\begin{definition}
    Let $l$ be a continuous linear functional on a Hilbert space $\mathcal{H}$. Then, there exists a unique $g\in\mathcal{H}$ such that
    \[l(f)=(f,g)\quad for\ all\ f\in\mathcal{H}\]
    Moreover, $||l||=||g||$
\end{definition}

\begin{proposition}
    Let $T:\mathcal{H}\to\mathcal{H} $ be a bounded linear transformation. There exists a unique bounded linear transformation $T^*$on $\mathcal{H} $ so that:
    \begin{description}
        \item[(i)] $(Tf,g)=(f,T^*,g)$.
        \item[(ii)] $||T||=||T^*||$.
        \item[(iii)] $(T^*)^*=T$
    \end{description}
    The linear operator $T^*:\mathcal{H}\to\mathcal{H}$ satisfting the above conditions is called the \textbf{adjoint} of $T$.
\end{proposition}

\begin{definition}
    Suppose $\{\varphi_k\}_{k=1}^\infty$ is an orthonormal basis of $\mathcal{H}$. Then, a linear transformation $T:\mathcal{H}\to\mathcal{H}$ is said to be diagonized with respect to the basis $\{\varphi_k\}$ If
    \[T(\varphi_k)=\lambda_k\varphi_k,\quad where\ \lambda_k\in\mathbb{C}\ for\ all\ k.\]
\end{definition}
\section*{References}
\end{document}